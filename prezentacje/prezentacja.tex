\documentclass{beamer}

\usepackage[utf8]{inputenc}
\usepackage{polski}
%\usepackage{titling}
\usepackage{romannum}
\usepackage{amsmath}
\usepackage{amssymb}
\usepackage{amsthm}
\usepackage{mathdots}
\usepackage{gensymb}
\usepackage{MnSymbol}
\usepackage{stmaryrd}

\def\R{\mathbb{R}}
\def\C{\mathbb{C}}
\def\Z{\mathbb{Z}}
\def\Q{\mathbb{Q}}
\def\N{\mathbb{N}}
\def\Rn{\mathbb{R}^n}
\def\E{\mathcal{E}}
\def\B{\mathcal{B}}
\def\nor{\trianglelefteq}
\def\ker{\operatorname{ker}}
\def\gengru#1{\langle\,#1 \,\rangle}
\def\ch{\blacktriangleleft}
\def\arr{\longrightarrow}
\def\Abs#1{\left\vert#1\right\vert}
\def\rk{\operatorname{rank}}
\def\lin{\operatorname{lin}}
\def\af{\operatorname{af}}
\def\dim{\operatorname{dim}}
\def\ker{\operatorname{ker}}
\def\im{\operatorname{im}}
\def\tr{\operatorname{tr}}
\def\Hom{\operatorname{Hom}}
\def\Aut{\operatorname{Aut}}
\def\id{\triangleleft}
\def\iif{\operatorname{if}}
\newcommand{\norm}[1]{\left\lVert#1\right\rVert}
\def\normsign{\|\cdot\|}
\newcommand{\series}[3]{\sum_{#1}^{#2}#3}
\newcommand{\colw}[1]{\column{#1\textwidth}}

\usetheme{Warsaw}

\title{Optymalizacja dawkowania leku przy leczeniu raka}
\author{Tomasz Kanas}

\begin{document}

\maketitle

\begin{frame}
  \frametitle{Problem}
  Lek zabija komórki raka, ale szkodzi też pacjentowi. Ponadto niektóre komórki mogą się uodpornić na lek. Kiedy i w jakich dawkach podawać lek, aby zmaksymalizować jego skuteczność?
  \begin{alertblock}{To niezbyt precyzyjne}
    Jak działa ten lek? Od czego zależy skuteczność? Co to znaczy skuteczność?
  \end{alertblock}
\end{frame}

\begin{frame}
  \frametitle{Bardziej formalnie}
  \begin{itemize}
  \item Pacjent ma pewną ilość komórek nowotworowych.
  \item Pewna część z tych komórek jest odporna na lek.
  \item Liczba komórek nowotworowych zmienia się w czasie.
  \item Tempo zmian zależy od dawkowania leku.
  \item Nie można przekroczyć maksymalnej dawki leku.
  \item Mamy pewną funkcję celu która dla danego dawkowania i przebiegu leczenia zwróci jak dobrze nam poszło.
  \end{itemize}
\end{frame}

\begin{frame}
  \frametitle{Problem optymalnego sterowania}
  \begin{center}Szukana jest funkcja $g : [0,T] \to \R$ taka, że: \end{center}
  \begin{columns}
    \colw{.5}
    \[\min_g J(V, g) \text{ gdzie }\]
    \[\dot{V}(t) = F(V, g, t),\; 0 \le t \le T\]
    \[V(0) = v_0\]
    \[\forall_t 0 \le g(t) \le c_{max}\]
    \colw{.5}
    \begin{itemize}
    \item $V$ --- stan układu
    \item $g$ --- sterowanie
    \item $F$ --- dynamika układu
    \item $J$ --- funkcja celu
    \end{itemize}
  \end{columns}
\end{frame}

\begin{frame}
  \frametitle{Model}
  \begin{columns}
    \colw{.6}
    \[V_1' = \lambda_1V_1F\left(\frac{V_1 + \alpha_{12}V_2}{K}\right) - \beta_1V_1g\]
    \[V_2' = \lambda_2V_2F\left(\frac{V_2 + \alpha_{21}V_1}{K}\right) - \beta_2V_2g\]
\[K' = -\mu K + (V_1+V_2) - d{(V_1 + V_2)}^{2/3}K - \beta K g\]
    \colw{.4}
    \begin{itemize}
    \item $g$ --- stężenie leku (sterowanie)
    \item $V_1$ --- Komórki guza podatne na lek
    \item $V_2$ --- Komórki guza odporne na lek
    \item $K$ --- Unaczynienie
    \item $J$ --- Funkcja celu
    \end{itemize}
  \end{columns}
\[J(V_1, V_2, K, g) = \int_0^T V_1(t) + V_2(t)dt + \omega\int_0^T G\left(\frac{V_2(t) - V_1(t)}{\epsilon}\right) dt \]
  \[F(x) = -\ln(x),\: G(x) = \frac{1+\tanh(x)}{2}\]
  \[\alpha, \beta, \lambda, \mu, d, \omega, \epsilon, g_{max}, T \text{ --- stałe (znane)} \]
\end{frame}

\begin{frame}
  \frametitle{Pierwsze spostrzeżenia}
  \begin{itemize}
  \item Problem optymalnego sterowania wygląda na bardzo ogólny i w ogólności trudny do rozwiązania.
  \item Model wygląda na dość skomplikowany.
  \item Obliczenie wartości $J$ dla danego sterowania wymaga rozwiązania skomplikowanego równania różniczkowego.
  \item Wartość $J$ zależy od funkcji, czyli zadanie optymalizacji jest nieskończenie wymiarowe.
  \end{itemize}
  \begin{block}{Wniosek}
    Musimy zadowolić się rozwiązaniem przybliżonym. Zdefiniujmy więc problem przybliżony.
  \end{block}
\end{frame}


\begin{frame}
  \frametitle{Pomysł}
  \begin{center}{\large Dyskretyzacja czasu}\end{center}
  \begin{itemize}
  \item Weźmy pewien skończony zbiór punktów w czasie:
    \[0 = t_0 < t_1 < \cdots < t_n = T\]
  \item Będziemy przybliżać sterowanie funkcją kawałkami stałą z węzłami w wybranych punktach.
  \item Mając takie sterowanie możemy numerycznie obliczyć przybliżone rozwiązanie równania róźniczkowego, oraz obliczyć funkcję celu.
  \end{itemize}
  \begin{block}{Podsumowując}
    Musimy już tylko zapisać funkcję celu w zależności od wartości na przedziałach i znaleźć minimum.
  \end{block}
\end{frame}

\begin{frame}
  \frametitle{Optymalizacja nieliniowa z ograniczeniami}
  \begin{columns}
    \colw{.5}
    \[\min_x J(x) \text{ z ograniczeniami }\]
    \[f(x) = 0\]
    \[h(x) <= 0\]
    \[\forall_i 0 \le x_i \le x_{max}\]
    \colw{.5}
    \begin{itemize}
    \item $J$ --- funkcja celu
    \item $f$ --- ograniczenia równościowe
    \item $h$ --- ograniczenia nierównościowe
    \end{itemize}
  \end{columns}
  \begin{block}{Spostrzeżenie}
    Ten problem, jak wiemy, nadal nie jest prosty w rozwiązaniu, ale przynajmniej istnieją gotowe biblioteki które mogą rozwiązać go za nas.
  \end{block}
\end{frame}

\begin{frame}
    \frametitle{Ostateczna forma zadania}
    \[\min_{x_1,\ldots,x_n} \hat{J}(x_1,\ldots, x_n) \text{ z ograniczeniami } \]
    \[\forall_i 0 \le x_i \le x_{max}\]
    Gdzie $x_i$ to przybliżona wartość sterowania na przedziale $[t_{i-1}, t_i]$, $\hat{J}$ to numeryczne przybliżenie funkcji celu.
\end{frame}


\begin{frame}
  \frametitle{Plan rozwiązania}
  \begin{itemize}
  \item Aproksymacja prowadząca do zadania optymalizacji nieliniowej w przestrzeń skończonego wymiaru.
  \item Implementacja w MATLAB-ie z wykorzystaniem gotowych narzędzi do optymalizacji nieliniowej.
  \item Znalezienie parametrów przy których optymalizacja zbiega i daje możliwie dobry wynik.
  \item Testy numeryczne metody --- weryfikacja.
  \item Opis rozwiązania wraz z motywacją dokonanych wyborów, dyskusją i krytyką otrzymanych wyników.
  \end{itemize}
\end{frame}

\begin{frame}
  \frametitle{Możliwe problemy}
  \begin{itemize}
  \item Problem skończonego wymiaru może okazać się za duży i skomplikowany.
  \item Interesuje nas minimum globalne, a narzędzia znajdują zwykle minimum lokalne, więc trzeba znaleźć dobry punkt startowy, lub użyć innych algorytmów.
  \item Optymalne sterowanie może być nieciągłe, co może być źródłem błędów przybliżenia i powodować konieczność zagęszczenia siatki dyskretyzacji.
  \item Jak mierzyć poprawność otrzymanego wyniku?
  \end{itemize}
\end{frame}

\begin{frame}
  \frametitle{Pomysły na zwiększenie rzędu aproksymacji}
  \begin{itemize}
  \item Przybliżanie splajnem (funkcją ciągłą, kawałkami wielomianową), a nie tylko funkcją kawałkami stałą.
  \item Implementacja algorytmu szukającego odpowiednio gęstego zbioru punktów i odpowiednich stopni wielomianów na przedziałach.
  \item Uruchomienie rozwiązania dla wielu punktów startowych w celu znalezienia najlepszego.
  \item Dostosowanie algorytmów optymalizacji nieliniowej do tego zadania.
  \item Wyliczenie różnych (dostępnych w literaturze) miar błędu i analiza wpływu parametrów na nie.
  \end{itemize}
\end{frame}

\begin{frame}
  \center{\Huge Dziękuję za uwagę}
\end{frame}

\end{document}
