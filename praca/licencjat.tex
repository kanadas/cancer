\documentclass[11pt]{article}

\usepackage[utf8]{inputenc}
\usepackage[margin=1cm]{geometry}
\usepackage{polski}
\usepackage{titling}
\usepackage{romannum}
\usepackage{amsmath}
\usepackage{amssymb}
\usepackage{amsthm}
\usepackage{mathdots}
\usepackage{fullpage}
\usepackage{gensymb}
\usepackage{MnSymbol}
\usepackage{stmaryrd}

\def\R{\mathbb{R}}
\def\C{\mathbb{C}}
\def\Z{\mathbb{Z}}
\def\Q{\mathbb{Q}}
\def\N{\mathbb{N}}
\def\Rn{\mathbb{R}^n}
\def\E{\mathcal{E}}
\def\B{\mathcal{B}}
\def\nor{\trianglelefteq}
\def\ker{\operatorname{ker}}
\def\gengru#1{\langle\,#1 \,\rangle}
\def\ch{\blacktriangleleft}
\def\arr{\longrightarrow}
\def\Abs#1{\left\vert#1\right\vert}
\def\rk{\operatorname{rank}}
\def\lin{\operatorname{lin}}
\def\af{\operatorname{af}}
\def\dim{\operatorname{dim}}
\def\ker{\operatorname{ker}}
\def\im{\operatorname{im}}
\def\tr{\operatorname{tr}}
\def\Hom{\operatorname{Hom}}
\def\Aut{\operatorname{Aut}}
\def\id{\triangleleft}
\def\iif{\operatorname{if}}
\newcommand{\norm}[1]{\left\lVert#1\right\rVert}
\def\normsign{\|\cdot\|}
\newcommand{\series}[3]{\sum_{#1}^{#2}#3}

\setlength{\droptitle}{-2cm}
\title{Optymalna strategia podawania leku}
\author{Tomasz Kanas}

\begin{document}
\pagenumbering{gobble}
\maketitle

\section{Sformułowanie problemu}

Celem pracy jest znalezienie strategii podawania leku, przy leczeniu nowotworu, pozwalającej osiągnąć możliwie największą skuteczność terapii. W tym celu skorzystamy z modelu przedstawionego w pracy [tu wypada zacytować wyjściową pracę]. Model ten przedstawia rozwój nowotworu w czasie w zależności od dawki leku za pomocą równania różniczkowego:
\begin{equation} \label{ode}
  \begin{aligned} 
    V_1'(t) &= \lambda_1V_1F\left(\frac{V_1 + \alpha_{12}V_2}{K}\right) - \beta_1V_1g(t), \\
    V_2'(t) &= \lambda_2V_2F\left(\frac{V_2 + \alpha_{21}V_1}{K}\right) - \beta_2V_2g(t), \\
    K'(t) &= -\mu K + (V_1+V_2) - d{(V_1 + V_2)}^{2/3}K - \beta K g(t) \\
  \end{aligned}
\end{equation}
dla $t \in [0, T]$, z warunkami początkowymi
\begin{equation} \label{ode-start}
   V_1(0) = V_{10},\ V_2(0) = V_{20},\ K(0) = K_0
\end{equation}
gdzie $F(x) = -\ln(x)$, $ 0 \le g(t) \le g_{\max}$, oraz $\lambda_1, \lambda_2, \ldots$ są zadanymi parametrami

Funkcja $V_1(t)$ modeluje liczbę komórek guza podatnych na lek w momencie $t$, $V_2(t)$ liczbę komórek guza odpornych na lek, a $K(t)$ jest parametrem nazwanym w pracy ``uczynnieniem''.

Zadanie polega na znalezieniu mierzalnej funkcji $g: [0, T] \to [0, g_{\max}]$ takiej, że rozwiązanie (\ref{ode}) minimalizuje funkcjonał
\begin{equation} \label{objf}
  J(V_1, V_2, K) = \int_0^T V_1(t) + V_2(t)dt + \omega\int_0^T G\left(\frac{V_2(t) - V_1(t)}{\epsilon}\right) dt
\end{equation}
gdzie
\begin{equation*}
  G(x) = \frac{1+\tanh(x)}{2} \quad
  \omega, \epsilon > 0 
\end{equation*}
%Parametrem równania różniczkowego (\ref{ode}) jest funkcja $g: [t_0, T] \to [0, g_{\max}]$. Jeśli dla danej funkcji $g$ równanie (\ref{ode}) posiada rozwiązanie, to możemy przypisać jej wartość funkcjonału $J$ dla tego rozwiązania. Oznacza to, że możemy przyjąć $J(g) := J(V_1, V_2, K)$, gdzie $V_1, V_2, K$ spełniają równanie (\ref{ode}). Dzięki temu naszym celem staje się znalezienie takiej funkcji $g$ dla której istnieje rozwiązanie równania (\ref{ode}) oraz wartość $J(g)$ jest minimalna.

Problem ten w literaturze nazywa się problemem optymalnego sterowania, a funkcję $g$ sterowaniem.

Zauważmy, że nie wymagamy of $g$ nawet ciągłości, więc rozwiązanie (\ref{ode}) może nie istnieć. Oczywiście w zadaniu interesują nas tylko takie $g$ dla których rozwiązanie (\ref{ode}) istnieje i jest jednoznaczne, więc można sterowaniu przydzielić wartość funkcjonału $J(g) = J(V_1, V_2, K)$. Zauważmy też, że funkcja $F$ posiada osobliwość w 0, ale nie jest to dla nas problemem, ponieważ $\alpha_{12},\alpha_{21} > 0$, więc osiągnięcie tej osobliwości nastąpi tylko gdy $V_1 = V_2 = 0$, czyli gdy pacjent nie ma żadnych komórek nowotworowych, więc jest wyleczony i możemy zakończyć terapię.

Podsumowując, celem pracy jest znalezienie funkcji mierzalnej $g : [0, T] \to [0, g_{\max}]$, oraz funkcji $V_1, V_2, K : [0, T] \to [0, \infty)$ spełniających (\ref{ode}) i (\ref{ode-start}), oraz minimalizujących funkcjonał celu (\ref{objf}).  

\subsection{Problem przybliżony}
Analityczne rozwiązywanie problemu optymalnego sterowania rzadko kiedy jest możliwe, a nawet gdy jest możliwe, jest trudne. Z tego powodu zdecydujemy się na szukanie rozwiązania przybliżonego.

Najpierw ograniczmy problem do problemu optymalizacji skończenie wymiarowej. W tym celu wprowadźmy dyskretyzacje czasu, poprzez ustalenie siatki dyskretyzacji, czyli ciągu punktów w czasie:
\begin{equation}
  0 = t_0 < t_1 < \cdots < t_{n-1} < t_n = T\]
\end{equation}
%Możemy teraz przybliżać sterowanie za pomocą splajnu z węzłami w punktach $t_i$. Ustalmy stopień splajnu na $i$-tym przedziale na $s_i$ i przedstawmy przybliżone sterowanie $\hat{g}$ na przedziale $[t_{i-1}, t_i)$ za pomocą bazy Lagrange'a opartej na węzłach równoodległych $t_{ij} = t_{i-1} + h_0j$, $n_i = \frac{t_i - t_{i-1}}{h_0}$:
%\begin{equation} \label{control}
%  \hat{g}(t) = \sum_{j = 0}^{s_i}g_{ij}l_{ij}(t),\quad l_{ij}(t) = \frac{\prod_{k=0,k\neq j}^{n_i} (t - t_{ik})}{\prod_{k=0,k\neq j}^{n_i} (t_{ij} - t_{ik})}
%\end{equation}
%Dzięki wyborowi bazy Lagrange'a możemy łatwo .
Możemy teraz przybliżać sterowanie za pomocą funkcji kawałkami stałej:
\begin{equation} \label{control}
  \hat{g}(t) = g_i \text{ gdy } t \in [t_{i-1}, t_i)
\end{equation}
Zauważmy, że przy takim sterowaniu prawa strona (\ref{ode}) nie jest ciągła, ale jest różniczkowalna i Lipszycowska na każdym przedziale, więc z tw. Picarda-Lindel\"ofa o istnieniu i jednoznaczności rozwiązań, dla dowolnego punktu początkowego rozwiązanie (\ref{ode}) na każdym przedziale $[t_{i-1}, t_i]$ istnieje i jest jednoznaczne. Dla pierwszego przedziału punkt początkowy mamy zadany przez (\ref{ode-start}), dla $i$-tego przedziału punkt początkowy jest jednoznacznie zadany przez rozwiązanie dla poprzedniego przedziału, więc rozwiązanie na całym $[0, T]$ jest określone jednoznacznie.

Mając tak przybliżoną funkcję sterowania możemy numerycznie przybliżyć rozwiązanie równania różniczkowego (\ref{ode}). Użyjemy do tego metody Runggego-Kutty rzędu $r$ ze stałym krokiem długości $h$. Dla uproszczenia notacji oznaczmy (\ref{ode}) przez
\begin{equation} \label{ode-sim}
  \dot{y}(t) = f(t, y, g),\ y = \begin{pmatrix} V_1 \\ V_2 \\ K \end{pmatrix},\ f = \begin{pmatrix} f_1 \\ f_2 \\ f_3 \end{pmatrix}
\end{equation}
wtedy przybliżone rozwiązanie (\ref{ode-sim}) na przedziale $[t_n + ah, t_n + (a + 1)h]$ wyraża się przez:
\begin{equation} \label{rk}
  \begin{split}
    &k_1 = f(t_n + ah, \hat{y}(t_n + ah), \hat{g}) \\
    &k_l = f(t_n + c_l h, \hat{y}(t_n + ah) + h \sum_{i = 1}^{l-1} a_{li}k_i, \hat{g}) \\
    &\hat{y}((a+1)h) = \hat{y}(ah) + h \sum_{i = 1}^r b_i k_i
  \end{split}
\end{equation}
gdzie $c_l, a_{li}, b_i$ są stałymi zależnymi od wybranej metody.

Zostało już tylko przybliżyć funkcjonał celu (\ref{objf}). Zapiszmy go w postaci
\begin{equation} \label{objf-sim}
  J(y) = \int_0^T j(y(t)) dt
\end{equation}
gdzie
\begin{equation}
  j(y(t)) = V_1 + V_2 + \omega G\left(\frac{V_2 - V_1}{\epsilon} \right),\ y(t) = \begin{pmatrix} V_1 \\ V_2 \\ K \end{pmatrix}
\end{equation}
wtedy ogólny wzór na kwadraturę ze stałym krokiem $h_1$ przybliżającą (\ref{objf-sim}) to
\begin{equation} \label{quad}
  \hat{J}(\hat{y}) = \sum_{i = 1}^N \alpha_i j(\hat{y}(ih))
\end{equation}
gdzie $N = \frac{T}{h}$, a $\alpha_i$ są stałymi zależnymi od wybranej kwadratury.

W ten sposób wyraziliśmy przybliżoną funkcję celu jako funkcję $g_1,\ldots,g_n$, więc problem przybliżony sprowadza się do
\begin{equation} \label{nlp}
  \begin{split}
    \min_{g_1,\ldots,g_n} &\hat{J}(g_1,\ldots, g_n) \text{ z ograniczeniami } \\
    &\forall_{i \in \{1,\ldots,n\}} 0 \le g_i \le g_{\max}
  \end{split}
\end{equation}

Jest to problem optymalizacji nieliniowej z ograniczeniami i istnieją implementacje metod pozwalających uzyskać przybliżone rozwiązanie tego problemu. To podejście do numerycznego problemu optymalnego sterowania jest podobne do zaproponowanego w~\cite{diehl} i nazywa się sekwencyjnym (ang. ``sequential''). 

\subsection{Plan rozwiązania}
Aby obliczyć wynik problemu przybliżonego skorzystamy ze środowiska MatLab/Octave wraz z dostarczonym z nim optymalizatorem problemu optymalizacji nieliniowej z ograniczeniami (odpowiednio FMINICON w MatLab, NONLIN\_MIN w Octave). W tym celu zaimplementujemy przejście od problemu optymalnego sterowania. Będziemy też musieli znaleźć odpowiednią siatkę dyskretyzacji i punkt startowy dla optymalizatora.

\section{Implementacja}
Optymalizator NONLIN\_MIN w Octave posiada prosty interfejs, jest to funkcja której należy dostarczyć funkcję do zoptymalizowania, punkt startowy, ograniczenia oraz ewentualne opcje dodatkowe.

Zauważmy, że znalezienie dobrego wyniku będzie wymagało zapewne wielokrotnego wywołania naszej funkcji celu przez optymalizator, więc zależy nam aby funkcja celu liczyła się możliwie szybko. Lepsza wydajnościowo implementacja pozwoli też na większe zagęszczenie siatki dyskretyzacji i tym samym wzrost dokładności aproksymacji.

Jedną z praktyk pozwalającą poprawić wydajność programów w środowiskach MatLab i Octave jest tak zwana wektoryzacja. Polega ona na zastępowaniu pętli operacjami na wektorach. Korzysta to z faktu, że wiele funkcji w tych środowiskach można wywołać z wektorem parametrów, zamiast pojedynczego parametru i zwracają one wtedy wektor wyników, oraz wykonują się znacznie szybciej niż gdyby wywołać je wielokrotnie w pętli. Z tego powodu będziemy korzystać z tej techniki gdzie to tylko możliwe.

Aby sprowadzić problem optymalnego sterowania do problemu optymalizacji nieliniowej musimy przybliżyć rozwiązanie równania różniczkowego (\ref{ode}) a następnie za pomocą uzyskanego rozwiązania przybliżyć wartość funkcji celu (\ref{objf}).

\subsection{Równanie różniczkowe}
Do przybliżenia równania (\ref{ode}) skorzystamy z dostarczonej w Octave implementacji metody Dormanda-Prince'a 4-tego rzędu ze zmiennym krokiem. Jest to popularna metoda pozwalająca uzyskać wysoką dokładność, ponadto umożliwia przekazanie wektora punktów w czasie w których chcemy otrzymać przybliżenie wyniku. Pozwala to rozwiązywać równanie w sposób zwektoryzowany i okaże się bardzo wygodne przy implementacji funkcji celu.

\subsection{Funkcja celu}
Aby obliczyć wartość funkcji celu użyjemy dostarczonej w Octave implementacji adaptacyjnej kwadratury Gaussa-Konroda. Osiąga ona dobrą dokładność, ewaluując odcałkowywaną funkcję w dość niewielu punktach. Zakłada też, że odcałkowywana funkcja jest zwektoryzowana, co wykorzystamy do poprawy wydajności rozwiązania.

Pewną niedogodnością jest fakt, że nie możemy nic zakładać o kolejności argumentów w wektorze który kwadratura przekazuje odcałkowywanej funkcji, natomiast funkcja rozwiązujące równanie różniczkowe wymaga by punkty były jej przekazane w kolejności rosnącej. Musimy więc posortować wektor argumentów przed przekazaniem ich do funkcji liczącej rozwiązanie równania różniczkowego, oraz przywrócić pierwotny porządek dla wyników tych obliczeń.

\section{Eksperyment}
\begin{align*}
  t_0 &= 0 & \alpha_{12} &= ?? \text{(przyjąłem 0.1)}  & \beta_1 &= 0.15 & \lambda_1 &= 0.192\\
  T &= 200 & \alpha_{21} &= ?? \text{(przyjąłem 0.15)} & \beta_2 &= 0.1  & \lambda_2 &= 0.192\\
  g_{\max} &= 3 &&     d &= 0.00873                    & \beta &= 0.05  & \mu &= 0 \\
  V_{10} &= 20 &  V_{20} &= 280                        & K_0 = 650 \\
  &&       & \omega &= ?? \text{(przyjąłem 1000)} & \epsilon &= 0.01 \\
\end{align*}

\bibliography{bibliography}{}
\bibliographystyle{abbrv}
\end{document}
